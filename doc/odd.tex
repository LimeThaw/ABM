% How to compile:

% pdflatex odd
% bibtex odd
% pdflatex odd
% pdflatex odd

\documentclass{article}
\usepackage{listings}
\usepackage{url}
\title{ODD description of our Agent Based Model}
\author{Tierry H\"ormann \and Timo Laudi}

\begin{document}

	\maketitle

	\section{Purpose}
		The purpose of the model is to model human behavior in regard to ciminal activity involving
		firearms and to help identify ways in which we can reduce the number of incidences of such
		activity.
		We hypothesize that increasing the penalties for violent crimes involving firearms and the
		restriction of firearm aquisition and possession would lower the rate of violent crimes
		involving guns.

	\section{Entities, state variables, and scales} \label{entities}
		The entities in our simulation are people. They are characterized by the following attributes:

		\begin{itemize}
			\item Moral:\\
				This parameter models the moral and social values held by the agent.
			\item Enthusiasm:\\
				The higher the enthusiasm the more likely the agent is to commit a crime and the
				more risk he is going to take.
			\item Gun possession:\\
				Indicates if the agent owns a gun. This is considered in the agents' decision
				making and crucial for testing our hypothesis.
			\item Age:\\
				The older a person is, the less likely they are to change their general opinion.
				We model this by limiting the change in attributes like moral for agents depending
				on their age.
			\item Happiness:\\
				The main optimization parameter of the agent - He will try to maximize it (via
				crimes). May be modelled  with other, more subtle emotions in a later stage.
			\item Wealth:\\
				An agent with little wealth is more inclined to commit a crime that helps him
				increase this attribute.
		\end{itemize}

		Crime motivating attributes are Happiness (and/or other emotions) and Wealth. These are
		considered by the agent for its decisions. The attributes
		moral, enthusiasm, happiness and wealth are perturbed randomly on a small scale to
		approximate daily events that take small effect on the attributes, but would exceed the
		complexity of the simulation.
		The connectedness of the agent within the graph is also important for the simulation.
		However, since it can be computed from the graph we do not consider it as an inherent
		attribute of the agent.
		\par
		The agents are simulated within a social network with undirected, weighted edges.
		The edges represent aquaintances and the weights how well the agents know each other,
		or how close they are.
		All agents together are supposed to approximate a small town or similar social group or
		structure. Each simulation epoch corresponds to a single day, and the whole simulation
		runs for 10 years.
		\\\par
		TODO: Explain the constraints and interactions.

	\section{Process overview and scheduling}
		Time is modeled in discrete steps of one day. For each day, every agent (that is alive)
		updates its attributes based on the state of the simulation at the end of the last day.
		It then decides whether to commit a crime, and if so, what kind of crime to commit. based
		on that decision it then goes on to execute the crime.
		\par
		The updates for each day are divided into three stages: First, crimes are chosen and
		executed by the agents. Second the social network is updated: People disappear or are
		introduced to the network and connectedness and edge values are updated. In the last
		stage, agent attribute updates unrelated to the execution of crimes are performed.

		\begin{lstlisting}[caption=The daily global update routine, label=l_global_update]
	fun updateWorld() {

		for agent in network {
			decision := agent.chooseCrime()
			if decision /= NO_CRIME {
				agent.executeCrime(decision)
			}
		}

		network.updateNodes()

		for agent in network {
			agent.updateAttributes()
		}

	}
		\end{lstlisting}

	\section{Design principles}
		\subsection{Basic Principles}
			We have considered several theories and models from sociology, and tried to incorporate
			their core aspects into a decision making scheme for our agents. These are:
			\begin{itemize}
				\item Rational choice theory:\\
					\verb|https://books.google.ch/books?hl=en&lr=&id=QaUgne7fgYUC&oi=fnd&pg=PA126&dq=Rational+choice+theory&ots=2zRVUwh29g&sig=KwCK6O8B1nuldtPBOVwF5sOVcWs#v=onepage&q=Rational%20choice%20theory&f=false|
					Also: Social exchange theory:\\
					\verb|http://www.annualreviews.org/doi/pdf/10.1146/annurev.so.02.080176.002003|

				\item Social learning theory:\\
					\verb|https://books.google.ch/books?hl=en&lr=&id=6vnOPg_tpBUC&oi=fnd&pg=PA106&dq=Social+learning+theory&ots=hY9zMSEr64&sig=iOYC2COKt8159Ezojs2RcQrIQj8#v=onepage&q=Social%20learning%20theory&f=false|
					Moral values propagating within the graph (and maybe criminal records
					influence?)

				\item Social control theory:\\
					Modeled with moral and partially with punishment and expectations.
			\end{itemize}
			\par
			TODO: Maybe find more theories, describe them further and how they are used in our model.
			\par
			TODO: How do moral values spread?

		\subsection{Emergence}
			The most important output of the simulation is the rate of violent crimes involving
			firearms. Thus it is naturally meant to be emergent. The gun possession rate and
			wealth is part
			of the simulation input, and chosed in accordance with statistics. Since gun
			possession is never changed and wealth only changed randomly, none of them are
			emergent.
			The age parameter changes in natural, linear fashion independent of the state of the
			simulation, so it is not emergent either.
			\par
			Other parameters imposed based on statistics will be migration patterns (people
			moving) and birth and death
			rates. Moral, enthusiasm and happiness are normalized to a "normal" value, since
			we don't have a good metric or statistic for them.

		\subsection{Adaptation}
			Primarily the moral ideas of individuals vary depending on their neighbors in the
			social graph. The idea is to mimic the emergence of social values and principle
			within real societies. Since the weight of the edges determines how close two adjacent
			agents are, an agent's moral is influenced more by neighbors with a higher valued
			edge in between.
			Also the execution of a crime has an effect on the attributes
			of all participating agents.
			\\\par
			TODO: Add specific rules, maybe additional adapting attributes, add algo for
				moral propagation
			\par
			TODO: How are the crimes selected, how do they influence attributes?

		\subsection{Objectives}
			As mentioned in section \ref{entities}, we define a specific subset of the attributes as crime
			motivating attributes (CMA). These are attributes that the agents want to optimize. As
			a measure of the goodness of a state we compute a weighted sum of the values of these
			attributes. When deciding if it wants to commit a crime, an agent predicts the effect
			of its options on its CMA, computes the resulting sum and compares it to its current
			CMA sum and that of the other options. Based on that comparison it decides to do
			nothing or to commit a crime.
			\\\par TODO: Give specific formula for weighted sum

		\subsection{Learning}
			We currently do not have an explicit learning algorithm. However,
			some attributes of the agents (especially moral) are altered after a crime is executed,
			according to the result. Since some attributes (like moral) are CMAs, this can be
			regarded as an indirect and implicit learning algorithm.
			We plan to implement an explicit learning algorithm for our agents, maybe using a
			neural network approach.
			\\\par
			TODO: Find learning algorithm, record crimes (neural network?)

		\subsection{Prediction}
			-> See code - Predict outcome based on CMA values

		\subsection{Sensing}
			The agents do not perceive any state of the physical world, since it is not part of our
			model. They do, however, perceive the moral values of their neighbors in the social
			graph. The graph structure is primarily imposed as starting condition, but can change
			in the course of the simulation. The edges of the graph are weighted, and the weight
			indicates the intimacy of the connection between two agents. The higher this value is,
			the more are the agents influenced by each other's moral values. The idea is that
			people have more trust towards people they know well, and that interactions are usually
			more frequent between close friends or family members.

		\subsection{Interaction}
			The moral values of the agents are influenced by those of their neighbors. Apart from
			that Agents interact when someone commits a crime, including the neighbors of the victim.
			\par
			TODO: Describe these interactions in more detail.

		\subsection{Stochasticity} \label{stoch}
			We assume that the selection of a victim occurs randomly within the graph.
			The entity attributes are also chosen from random distributions according to
			statistical data or, where such is not available, from adequate distributions
			around a chosen "normal" value. During the simulation, the projected and actual
			outcome of crimes is varied randomly to account for the complexity of human thought
			and decision processes and unpredictable events during crime execution. The random
			numbers are generated using linear congruence. Where a normal distribution is used
			we calculated the random number from two uniformly random numbers using the
			Box-M\"uller transform \cite{boxmuller}.

		\subsection{Collectives}
			No collectives like families, social circles, gangs etc. are explicitly modeled,
			although they might emerge during simulation.

		\subsection{Observation}
			The main output data of the simulation is the rate of criminal activity. We divide
			those incidents into categories. The data is the output to a file in the form of a
			list of tuples, where each list entry carries the data for a single day, and the
			values in the tuples represent the different categories. For better visualization and
			analyzation they are then rendered into figures and processed into tables using a
			seperate program.
			\par
			Following are the main categories we divide our crimes in:
			\begin{itemize}
				\item Murders and manslaughter
				\item Violent assault
				\item Robbery, burglary, theft etc.
			\end{itemize}
			We also check for each instance if a firearm has been involved, and divide them
			accordingly.

	\section{Initialization}
		For the initialization we took a social network from the Stanford Large Network Dataset
		Collection \cite{snapnets} and added our agents with randomly assigned attributes (see
		section \ref{stoch}) to the nodes. The benefit of this is that we have a realistic
		social network for our simulation.

	\section{Input data}
		Our network changes during simulation. Nodes are added or deleted to account for
		people moving to or away from our simulated town and people dying or being born. To keep
		these changes realistic we collected statistical data for these events and made our model
		behave accordingly.
		\\\par
		TODO: Find the data, cite it and describe the algorithm for adding/deleting

	\section{Submodels}
		TODO: Describe in detail the concepts from 'Process overview and scheduling' - add
		Pseudocode and literature references. -> See code

	\medskip

	\bibliographystyle{apalike}
	\bibliography{references}{}

\end{document}