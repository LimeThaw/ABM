\documentclass{article}
\title{ODD description of the model}

\begin{document}

	\maketitle

	\section{Purpose}
		The purpose of the model is to model human behavior in regard to ciminal activity involving
		firearms and to help identify ways in which we can reduce the number of incidences of such
		activity.\\
		TODO: Get more precise $\Rightarrow$ hypothesis?

	\section{Entities, state variables, and scales}
		The entities in our simulation are people. They are characterized by the following attributes:

		\begin{itemize}
			\item Moral
			\item Enthusiasm
			\item Gun possession
			\item Age
		\end{itemize}

		All agents together are supposed to approximate a small town or similar social group or structure.
		Each simulation epoch corresponds to a single day, and the whole simulation runs for 10 years.\\
		TODO: Describe each attribute, explain the constraints, interactions, computed attributes.

	\section{Process overview and scheduling}
		Time is modeled in discrete steps of one day. For each day, every agent (that is alive)
		updates its attributes based on the state of the simulation at the end of the last day.
		It then decides wether to commit a crime, and if so, what kind of crime to commit. based
		on that decision it then goes on to execute the crime.\\
		TODO: Better description, Pseudocode (Maybe later when model is more worked out)

	\section{Design principles}
		\subsection{Basic Principles}
			We have considered several theories and models from sociology, and tried to incorporate
			their core aspects into a decision making scheme for our agents. These are:
			\begin{itemize}
				\item Rational choice theory:\\
					\verb|https://books.google.ch/books?hl=en&lr=&id=QaUgne7fgYUC&oi=fnd&pg=PA126&dq=Rational+choice+theory&ots=2zRVUwh29g&sig=KwCK6O8B1nuldtPBOVwF5sOVcWs#v=onepage&q=Rational%20choice%20theory&f=false|
					Also: Social exchange theory:\\
					\verb|http://www.annualreviews.org/doi/pdf/10.1146/annurev.so.02.080176.002003|

				\item Social learning theory:\\
					\verb|https://books.google.ch/books?hl=en&lr=&id=6vnOPg_tpBUC&oi=fnd&pg=PA106&dq=Social+learning+theory&ots=hY9zMSEr64&sig=iOYC2COKt8159Ezojs2RcQrIQj8#v=onepage&q=Social%20learning%20theory&f=false|
			\end{itemize}
			TODO: Maybe find more theories, describe them further and how they are used in our model.\\
			TODO: How do moral values spread?

		\subsection{Emergence}
			The most important output of the simulation is the rate of violent crimes involving
			firearms. Thus it is naturally meant to be emergent. The gun possession rate is part
			of the simulation input and is not actively changed during it, which makes it imposed.
			The age parameter changes in natural, linear fashion independent of the state of the
			simulation, so it is not emergent either.\\
			TODO: Check all parameters, search for imposed ones to constrain others.

		\subsection{Adaptation}
			Primarily the moral ideas of individuals vary depending on their neighbors in the social
			graph. The idea is to mimic the emergence of social values and principle within real
			societies.\\
			TODO: Add specific rules, maybe additional adapting attributes

		\subsection{Objectives}
			TODO: Do our agents have specific objectives? Which ones?

		\subsection{Learning}
			TODO: Do our agents learn over time? If so how?

		\subsection{Prediction}
			TODO: How do our agents attempt to predict the outcome of their actions (if they do)?

		\subsection{Sensing}
			The agents do not perceive any state of the physical world, since it is not part of our
			model. They do, however, perceive the moral values of their neighbors in the social
			graph. The graph structure is primarily imposed as starting condition, but can change
			in the course of the simulation. The edges of the graph are weighted, and the weight
			indicates the intimacy of the connection between two agents. The higher this value is,
			the more are the agents influenced by each other's moral values. The idea is that
			people have more trust towards people they know well, and that interactions are usually
			more frequent between close friends or family members.

		\subsection{Interaction}
			The moral values of the agents are influenced by those of their neighbors. Apart from
			that Agents interact when someone commits a crime, including the neighbors of the victim.\\
			TODO: Describe these interactions in more detail.

		\subsection{Stochasticity}
			We assume that the selection of a victim occurs randomly within the graph.\\
			TODO: Add other random approximations

		\subsection{Collectives}
			TODO: Do we want to add some kind of collectives? Maybe too complex?

		\subsection{Observation}
			The main output data of the simulation is the rate of criminal activity. We divide
			those incidents into categories. The data is the output to a file in the form of a
			list of tuples, where each list entry carries the data for a single day, and the
			values in the tuples represent the different categories. For better visualization and
			analyzation they are then rendered intofigures and processed into tables using a
			seperate program.\\
			TODO: Add categories

	\section{Initialization}
		TODO: Is our data generated randomly, taken from a dataset or a hybrid?

	\section{Input data}
		TODO: Do we have or can we obtain data to feed into the simulation as it runs instead of
		simulating it?

	\section{Submodels}
		TODO: Describe in detail the concepts from 'Process overview and scheduling' - add
		Pseudocode and literature references.

\end{document}